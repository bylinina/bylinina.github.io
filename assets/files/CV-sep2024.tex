\documentclass[12pt,letterpaper]{article}
\usepackage[margin=1in]{geometry} % 1-inch margin
\usepackage{txfonts} % Times New Roman
%\pagestyle{empty} % No page number
\usepackage{linguex}\renewcommand{\firstrefdash}{}
\usepackage{parskip}\setlength{\parskip}{0pt}\setlength{\parindent}{1em}
\usepackage[normalem]{ulem}
\renewcommand{\baselinestretch}{.99}
\usepackage{braket}
\usepackage{mathabx}
\usepackage{multicol}
\usepackage{hanging}
\usepackage[official]{eurosym}
\usepackage{hyperref}

\hypersetup{
    colorlinks=true,
    linkcolor=blue,
    filecolor=magenta,      
    urlcolor=cyan,
    pdfpagemode=FullScreen,
    }

\urlstyle{same}

\setlength{\parindent}{12pt}

\begin{document}


{\centering
    \LARGE{Lisa Bylinina $\bullet$ Curriculum Vitae}
\\}

\vspace{6mm}

\noindent \Large{Contact Information}

\vspace{-4mm}
\noindent\line(1,0){250}

\small{
\noindent Phone: +31(0)618350454;  Email: \href{mailto://bylinina@gmail.com}{bylinina@gmail.com}, \href{mailto://e.g.bylinina@rug.nl}{e.g.bylinina@rug.nl}; Website: \href{https://bylinina.github.io}{bylinina.github.io}
}

\vspace{5mm}

\noindent \Large{Personal Information}

\vspace{-4mm}
\noindent\line(1,0){250}

\small{
\noindent {\bf Name as appears in passport}: Elizaveta Grigoryevna Bylinina\\
\noindent {\bf Date of birth}: 12 April 1983\\
\noindent {\bf Place of birth}: Vladivostok, USSR\\ 
\noindent {\bf Citizenship}: Russian Federation; The Netherlands\\ 
\noindent {\bf Current place of residence}: Hilversum, The Netherlands
%\noindent {\bf Family}: Not married. Daughter Vera, born 25 May 2008
}

\vspace{4mm}

\noindent \Large{Current employment}

\vspace{-4mm}
\noindent\line(1,0){250}


\small{
\noindent Assistant Professor (UD1) at Utrecht University\\Faculty of Humanities\\
Department of Languages, Literature and Communication \\
\href{https://www.uu.nl/en/research/institute-for-language-sciences}{Institute for Language Sciences}\\
\href{https://www.uu.nl/staff/organisationalchart/gw/149/1002}{Language and Communication Group}\\
\textbf{from September 2024}
}


\vspace{5mm}


\noindent \Large{Previous Research or Academic Positions}

\vspace{-4mm}
\noindent\line(1,0){290}

\vspace{2mm}

\small{
\noindent\textbf{December 2022 -- September 2024}
Assistant Professor (UD2)

Center for Language and Cognition, \href{https://www.rug.nl/research/clcg/research/cl/}{Computational Linguistics Group}

Faculty of Arts, University of Groningen
}

\vspace{2mm}
\small{
\noindent\textbf{July 2020 -- December 2022}
Machine learning and NLP engineer, Research and Development

\url{Bookarang.com}, Amsterdam
}

\vspace{2mm}
\small{
\noindent\textbf{January 2020 -- September 2020}
Associated researcher

Leiden University Centre for Linguistics, Leiden University

Leiden Institute for Brain and Cognition, Leiden University
}

\vspace{2mm}
\small{
\noindent\textbf{November 2016 -- January 2020}
Postdoctoral researcher

Principal investigator in NWO VENI project `Number Words'

Leiden University Centre for Linguistics, Leiden University
}

\vspace{2mm}

\small{
\noindent\textbf{November 2013 -- November 2016}

Postdoctoral researcher, NWO Horizon program `Knowledge and Culture'

The Meertens Instituut, the Royal Netherlands Academy of Arts and Sciences, Amsterdam
}

\vspace{2mm}

\small{
\noindent\textbf{November 2009 -- September 2013}

Ph.D. student, NWO project `Degrees under Discussion'

Utrecht Institute of Linguistics OTS, Utrecht University

Thesis title: `The Grammar of Standards: Judge-dependence, Purpose-relativity and Comparison Classes 

in Degree Constructions' (defended 14 Feb 2014)

Advisor: Rick Nouwen; promotor: Henri\"{e}tte de Swart
}

\vspace{2mm}

\noindent \Large{Grant history}

\vspace{-4mm}
\noindent\line(1,0){250}

\small{
\noindent Small Grant for Data Annotation, \euro{}500 
\vspace{1mm} \\
\noindent Funding source: Toloka.AI\\
\noindent August 2023
%\noindent Role in project: Principal Investigator\\
}
\vspace{3mm}

\small{
\noindent Start-Up Grant, \euro{}3k 
\vspace{1mm} \\
\noindent Funding source: The Faculty of Arts, University of Groningen\\
\noindent May 2023
%\noindent Role in project: Principal Investigator\\
}
\vspace{3mm}

\small{
\noindent NWO VENI grant for `Number Words' project, \euro{}243k 
\vspace{1mm} \\
\noindent Funding source: De Nederlandse Organisatie voor Wetenschappelijk Onderzoek (The Netherlands Organisation for Scientific Research)\\
\noindent Official project name: `Number Words' \\
%\noindent Grant period: November 2016 -- November 2019 \\
\noindent Role in project: Principal Investigator\\
\noindent November 2016
}
\vspace{3mm}

\small{
\noindent Digital Humanities Small Grant, \euro{}1.5k 
\vspace{1mm} \\
\noindent Funding source: Leiden University Centre for Digital Humanities \\
\noindent March 2019\\
%\noindent Role in project: Principal Investigator\\
}
\vspace{3mm}

\noindent \Large{Other Previous Employment}

\vspace{-4mm}
\noindent\line(1,0){250}

\small{
\noindent\textbf{March -- December 2009} Computational linguist 

Group of linguistic projects, NLP/IR 

Yandex, LLC, Moscow, Russia
}

\vspace{2mm}
%
%\small{
%\noindent\textbf{Summer 2008}
%
%Wrote a Russian syntax module for `AOT' (aot.ru), Moscow, Russia
%}
%
%\vspace{2mm}
%
\small{
\noindent\textbf{January 2007 -- January 2008}

Linguist / NLP project for Clarabridge at EPAM Systems, Moscow, Russia 
}

\vspace{2mm}

\small{
\noindent\textbf{October 2006 -- January 2007}

Linguist / Information retrieval at Smartware, Moscow, Russia
}
%
%\vspace{2mm}
%
%\small{
%\noindent\textbf{December 2005 -- March 2006}
%
%Linguist / Machine translation at ABBYY Software House, Moscow, Russia
%}
%
\vspace{2mm}

\small{
\noindent\textbf{April -- November 2005}

Editor / Linguistics department at Federal state unitary enterprise Great Russian Encyclopedia
}
%
\vspace{2mm}

\small{
\noindent\textbf{November 2002 -- January 2005}

Proofreader, then editor at Polit.ru, Moscow, Russia 
}

\vspace{5mm}




\noindent \Large{Education}

\vspace{-4mm}
\noindent\line(1,0){250}

\vspace{2mm}

\small{
\noindent\textbf{November 2009 -- September 2013}

Ph.D. student, Utrecht Institute of Linguistics OTS, Utrecht University

(Dissertation defended 14 Feb 2014)
}

\vspace{2mm}

\small{
\noindent\textbf{September --  December 2011}

Visit to UChicago Linguistics Department
}

\vspace{2mm}

\small{
\noindent\textbf{November 2005 -- Spring 2007 (interrupted)}

Graduate student in Institute of Linguistics (Russian Academy of Science)

Advisor: Ya. G. Testelets 
}

\vspace{2mm}

\small{
\noindent\textbf{September 2001 -- June 2005}

Theoretical and Applied Linguistics Department, 
Philological Faculty, Moscow State University

Diploma magna cum laudae (equiv. to MA with honours) 
in Theoretical and Applied Linguistics 

Advisors: K. I. Kazenin, Ya. G. Testelets 
}

\vspace{2mm}

\small{
\noindent\textbf{September 2000 -- June 2001 
(transferred to the Linguistics Department)}

Department of Russian Language and Literature, 
Philological Faculty, Moscow State University
}

\vspace{6mm}

\noindent \Large{Summer and Winter Schools} (as a student)

\vspace{-4mm}
\noindent\line(1,0){300}

\vspace{2mm}

\small{
\noindent\textbf{January 2022}

Advanced Language Processing School,  LIG (Univ. Grenoble Alpes) and Naver Labs Europe 
 
}

\vspace{2mm}

\small{
\noindent\textbf{December 2015}

 International School in Cognitive Sciences and Semantics, University of Latvia, Riga 
 
Intensive course `Number: Crosslinguistic and Crossdisciplinary Approaches'
}

\vspace{2mm}

\small{
\noindent\textbf{January 2012}

LOT winterschool, Tilburg, Netherlands}

\vspace{2mm}

\small{
\noindent\textbf{August 2011}

ESSLLI, Ljubljana, Slovenia
}

\vspace{2mm}

\small{
\noindent\textbf{June 2010}

LOT summerschool, Nijmegen, Netherlands
}

\vspace{2mm}

\small{
\noindent\textbf{June -- August 2005}

Linguistic Society of America Summer Institute, MIT \& Harvard University, Cambridge, MA

(Fellowship from Linguistic Society of America) 
}

\vspace{2mm}

\small{
\noindent\textbf{July 2004}

New York Institute of Cognitive and Cultural Studies, St. Petersburg, Russia
}

\vspace{2mm}

\small{
\noindent\textbf{Summer 2003}

Central European Summer School in Generative Grammar, Lublin, Poland
}

\vspace{2mm}

\small{
\noindent\textbf{Summer 2001}

Central European Summer School in Generative Grammar, Nis, Serbia
}

\vspace{5mm}

\noindent \Large{Dissertation}

\vspace{-4mm}
\noindent\line(1,0){250}


\small{
\begin{hangparas}{.25in}{1} 

Bylinina, Lisa (2014) {\it The Grammar of Standards: Judge-dependence, Purpose-relativity and Comparison Classes in Degree Constructions}. PhD dissertation. LOT Dissertation Series 347, 216 pages. ISBN: 978-94-6093-130-7.

\vspace{1mm}

\end{hangparas}
}

\vspace{5mm}

\pagebreak
\noindent \Large{Datasets}

\vspace{-4mm}
\noindent\line(1,0){250}

\small{
\begin{hangparas}{.25in}{1} 

Silvana Abdi, Hylke Brouwer, Martine Elzinga, Shenza Gunput, Sem Huisman, Collin Krooneman, David Poot, Jelmer Top, Cain Weideman, Lisa Bylinina (2024) {\it Dutch CoLA (Corpus of Linguistic Acceptability)} [\href{https://huggingface.co/datasets/GroNLP/dutch-cola}{HuggingFace page}]

\vspace{1mm}

\end{hangparas}
}

\small{
\begin{hangparas}{.25in}{1} 

Bylinina, Lisa and Sjef Barbiers (2019) {\it Numeral Typology Database}, DataverseNL. DOI: \url{http://doi.org/10411/VG1BZ8}

\vspace{1mm}

\end{hangparas}
}

\vspace{5mm}

\noindent \Large{In prep., under review / revision, accepted, preprints, manuscripts}

\vspace{-4mm}
\noindent\line(1,0){460}


\small{
\begin{hangparas}{.25in}{1} 

%Yasutada Sudo, Lisa Bylinina and Stavroula Alexandropoulou (in prep. for {\it Proceedings of ELM 3}). Priming acceptability judgments of NPI {\it any}. 

%\vspace{1mm} 
%
%Ivan Yamschikov, Alexey Tikhonov and Lisa Bylinina (under review) S-WINOGRAD: Multilingual Synthetic Winograd Schemata for Commonsense Reasoning.

\vspace{1mm} 

Jaap Jumelet, Lisa Bylinina, Willem Zuidema and Jakub Szymanik (under review for {\it TACL}). Black Big Boxes: Do Language Models Hide a Theory of Adjective Order? [\href{https://arxiv.org/abs/2407.02136}{preprint}]

\vspace{1mm} 

Lisa Bylinina (2023) Lecture notes: Linguistics for Language Technology. [\href{https://bylinina.github.io/ling_course/}{website}] 

\vspace{1mm} 

Lasha Abzianidze, Lisa Bylinina and Denis Paperno (in print, {\it Cambridge Elements in Semantics}) Semantics and deep learning. [\href{https://ling.auf.net/lingbuzz/007736}{preprint}]

\vspace{1mm} 

Alexey Tikhonov, Lisa Bylinina and Ivan Yamschikov (under review). Individuation in neural models with and without visual grounding.

\vspace{1mm} 

%Bylinina, Lisa and Heidi Klockmann (in prep. for {\it Language}) Review of Ionin and Matushansky (2019) Cardinals: The syntax and semantics of cardinal-containing expressions. The MIT Press.

\vspace{1mm} 

%Bylinina, Lisa and Sjef Barbiers (in prep.) Typology of numerals and the number line.
%
%\vspace{1mm} 

Blok, Dominique, Lisa Bylinina and Rick Nouwen (2018). Splitting Germanic negative indefinites. Ms. Longer version of 2017 Amsterdam Colloquium Proceedings paper.

\vspace{1mm} 

Bylinina, Lisa, Natalia Ivlieva, Alexander Podobryaev, Yasutada Sudo (2015). A Non-Superlative Semantics for Ordinals and the Syntax of Comparison Classes. Ms. Longer version of NELS proceedings paper.

%\vspace{1mm}
%
%Bylinina, Lisa (2015) The Structure of Inappropriateness: The Syntax and Semantics of the Attributive-with-Infinitive Construction. Under revision for {\it Natural Language and Linguistic Theory}. 

\vspace{1mm}


\end{hangparas}
}

\vspace{5mm}


\noindent \Large{Journal articles}

\vspace{-4mm}
\noindent\line(1,0){250}

\small{
\begin{hangparas}{.25in}{1} 

Dorison, C.A., Lerner, J.S., Heller, B.H. et al. [incl. Bylinina, L.] (2022). In COVID-19 Health Messaging, Loss Framing Increases Anxiety with Little-to-No Concomitant Benefits: Experimental Evidence from 84 Countries.  Affective Science 3, 577-602. \url{https://doi.org/10.1007/s42761-022-00128-3}

\vspace{1mm} 

Legate, N. et al. [incl. Bylinina, L.] (2022). A Global Experiment on Motivating Social Distancing during the COVID-19 Pandemic. Proceedings of the National Academy of Sciences 119.22. Preprint: \url{https://doi.org/10.31234/osf.io/n3dyf}

\vspace{1mm} 

Wang, Ke, et al. [incl. Bylinina, L.] (2021). A multi-country test of brief reappraisal interventions on emotions during the COVID-19 pandemic. {\it Nature Human Behaviour 5.8}: 1089-1110. \url{https://doi.org/10.1038/s41562-021-01173-x}

\vspace{1mm} 

Bylinina, Lisa and Rick Nouwen (2020). Numeral Semantics. {\it Language and Linguistics Compass}, 14:e12390. \url{https://doi.org/10.1111/lnc3.12390}

\vspace{1mm} 

Berghoff, Robyn, Rick Nouwen, Lisa Bylinina, and Yaron McNabb (2020). Degree modification across categories in Afrikaans. {\it Linguistic Variation} 20, no. 1 (2020): 102-135.  \url{https://doi.org/10.1075/lv.17004.ber}

\vspace{1mm} 

Bylinina, Lisa and Alexander Podobryaev (2019). Plurality in Buriat and Structurally Constrained Alternatives. {\it Journal of Semantics}. \url{https://doi.org/10.1093/jos/ffz017}

\vspace{1mm} 

Bylinina, Lisa and Rick Nouwen (2018). On `zero' and semantic plurality. {\it Glossa: a journal of general linguistics}, 3(1), p.98. \url{http://doi.org/10.5334/gjgl.441}

\vspace{1mm} 

Bylinina, Lisa (2017). Count lists cross-linguistically vs bootstrapping the counting system. {\it Snippets} 31. \url{http://doi.org/10.7358/snip-2017-031-byli} 

\vspace{1mm} 

Bylinina, Lisa (2017). Judge-dependence in degree constructions. {\it Journal of Semantics} 34.2, pp. 291--331. \url{http://doi.org/10.1093/jos/ffw011}

\vspace{1mm}

Bylinina, Lisa and Yasutada Sudo (2015). Varieties of intensification. {\it Natural Language and Linguistic Theory} 05/2015. \url{http://doi.org/10.1007/s11049-015-9291-y}

\vspace{1mm}


\end{hangparas}
}

\vspace{5mm}

\noindent \Large{Papers in volumes, collections and proceedings}

\vspace{-4mm}
\noindent\line(1,0){340}

\small{
\begin{hangparas}{.25in}{1} 

Bylinina, Lisa, Alexey Tikhonov \& Ekaterina Garmash. (2023). Connecting degree and polarity: An artificial language learning study. In {\it The 2023 Conference on Empirical Methods in Natural Language Processing} (main session).
\vspace{1mm} 

Edman, Lukas and Lisa Bylinina. (2023). Too Much Information: Keeping Training Simple for BabyLMs. In {\it Proceedings of the BabyLM Challenge at the 27th Conference on Computational Natural Language Learning} (pp. 89-97).

\vspace{1mm} 

Alexey Tikhonov, Lisa Bylinina, and Denis Paperno. (2023). Leverage Points in Modality Shifts: Comparing Language-only and Multimodal Word Representations. In {\it Proceedings of the 12th Joint Conference on Lexical and Computational Semantics (*SEM 2023)}, pp. 11-17, Toronto, Canada. Association for Computational Linguistics.

\vspace{1mm} 

Bylinina, Lisa and Alexey Tikhonov (2022). The driving forces of polarity-sensitivity: Experiments with multilingual pre-trained neural language models. {\it Proceedings of the Annual Meeting of the Cognitive Science Society, 44}. \url{https://escholarship.org/uc/item/9xj2t25t}.

\vspace{1mm} 

Bylinina, Lisa and Alexey Tikhonov (2022). Transformers in the loop: Polarity in neural models of language. {\it Proceedings of the 60th Annual Meeting of the Association for Computational Linguistics (Volume 1: Long Papers)}. \url{https://doi.org/10.18653/v1/2022.acl-long.455}
 
\vspace{1mm} 

Alexandropoulou, Stavroula, Lisa Bylinina and Rick Nouwen (2020). Is there {\it any} licensing in non-DE contexts? An experimental study. {\it Proceedings of Sinn Und Bedeutung}, 24(1), 35-47. \url{https://doi.org/10.18148/sub/2020.v24i1.851}

\vspace{1mm} 

Bylinina, Lisa and Alexander Podobryaev (2017). Plurality in Buriat and structurally constrained alternatives. {\it 21st Amsterdam Colloquium Proceedings.}

\vspace{1mm} 

Blok, Dominique, Lisa Bylinina and Rick Nouwen (2017). Splitting Germanic negative indefinites. {\it 21st Amsterdam Colloquium Proceedings.}

Bylinina, Lisa (2017). Sravnitelnye construkcii (Comparative constructions). {\it Elementy tatarskogo yazyka v tipologicheskom osveschenii} (Elements of Tatar language in a typological perspective). S.G. Tatevosov, A.G. Pazelskaya, Dzh. Sh. Suleymanov (eds.). Moscow. (In Russian)

\vspace{1mm}

Bylinina, Lisa, Natalia Ivlieva, Alexander Podobryaev, and Yasutada Sudo (2015).
An In Situ Semantics for Ordinals. In: Thuy Bui, Deniz \"{O}zy{\i}ld{\i}z (eds.), NELS 45: Proceedings of the Forty-Fifth Annual Meeting of the North East Linguistic Society: Volume 1, pp. 135-145.

\vspace{1mm}

Bylinina, Lisa, Elin McCready and Yasutada Sudo (2015) Notes on Perspective-Sensitivity. In: P. Arkadiev, I. Kapitonov, Y. Lander, E. Rakhilina, S. Tatevosov (eds.) 
Donum semanticum: Opera linguistica et logica in honorem Barbarae Partee a discipulis amicisque Rossicis oblata. / Yazyki slavianskoy kul'tury, Moscow. 

\vspace{1mm}

Bylinina, Lisa (2013). Degree infinitival clauses. {\it Proceedings of Semantics and Linguistic Theory} 23, pp. 294-411.

\vspace{1mm}

Bylinina, Lisa and Yuri Lander (2013). THAN = MORE + EVEN: Evidence from Kabardian. {\it Proceedings of Amsterdam Colloquium} 2013, pp. 75-82. 

\vspace{1mm}

Bylinina, Lisa (2013) Judge-dependence in degree constructions: Evidence from Japanese evidentiality. {\it Proceedings of Formal Approaches to Japanese Linguistics} 6 Cambridge: MITWPL \#66, pp. 17-28.

\vspace{1mm}

Bylinina, Lisa (2012). Functional standards and the absolute/relative distinction. In: A. Aguilar Guevara, A. Chernilovskaya, and R. Nouwen (eds.) Proceedings of Sinn und Bedeutung 16: Volume 1 (pp. 141-157) Cambridge: MITWPL. 

\vspace{1mm}

Bylinina, Lisa and Stas Zadorozhnyy (2012). Evaluative adjectives, scale structure, and ways of being polite. In: M. Aloni, V. Kimmelman, F. Roelfson, G. Sassoon, K. Schulz \& M. Westera (eds.) Logic, Language and Meaning: 18th Amsterdam Colloquium, Amsterdam, The Netherlands, December 2011 Revised Selected Papers. (pp. 133-142) Springer. 

\vspace{1mm}

Bylinina, Lisa (2011). `Functional' standard in Russian and English degree constructions. In: A.E. Kibrik, V.I. Belikov, I.M. Boguslawski, B.V. Dobrov, D.O. Dobrovolski \& D.O. Dobrovolski (eds.) Computational Linguistics and Intellectual Technologies Issue 10: Papers from the Annual International Conference 'Dialogue' (2011), pp. 161-168 Moscow: RGGU.

\vspace{1mm}

Bylinina, Lisa (2011). Functional standards. In: D. Lassiter (ed.), Proceedings of the 2011 ESSLLI Student Session (pp. 46-58). 

\vspace{1mm}

Bylinina, Lisa (2011) This is so NP! In: B.H. Partee, M. Glanzberg \& J. Skilters (eds.) FORMAL SEMANTICS AND PRAGMATICS. DISCOURSE, CONTEXT AND MODELS (pp. 1-15) Riga: New Prairie Press.

\vspace{1mm}

Bylinina, Lisa (2010) Wh-Reduplication in Altai. In: H. Maezawa \& A. Yokogoshi (eds.) Proceedings of the 6th Workshop on Altaic Formal Linguistics (WAFL6) Vol. 61. MIT working papers in linguistics. Cambridge: MIT.

\vspace{1mm}

Bylinina, Lisa (2010) Depreciative Indefinites: Evidence from Russian. In: G. Zybatow, P. Dudchuk, S. Minor \& E. Pshehotskaya (eds.) Formal Studies in Slavic Linguistics (Linguistik International), pp. 191-207. Frankfurt am Main: Peter Lang.

\vspace{1mm}

Bylinina, Lisa (2005) Ba\v{s}kirskij jazyk (Bashkir language). In: Bol'\v{s}aja Rossijskaya enciklopedia v 30 tomah (Great Russian Encyclopedia in 30 volumes). Volume 3. Moscow: BRE, p. 137.

\vspace{1mm}

Bylinina, Lisa and Yakov Testelets (2004) Novye neopredelennye mestoimenija v russkom jazyke / New indefinite pronouns in Russian. In: XXXIII Me\v{z}dunarodnaja filologi\v{c}eskaja konferencija. Vyp. 25. Sekcija prikladnoj I matemati\v{c}eskoj lingvistiki, tom 1. SPb: Philological faculty, SPbGU, pp. 40-46. 

\vspace{1mm}

Bylinina, Lisa and Yakov Testelets (2004) Sluicing-Based Indefinites in Russian. In: Formal Approaches to Slavic Linguistics 13: The South Carolina Meeting, pp. 355-365, ed. Steven Franks, Frank Y. Gladney and Mila Tasseva-Kurktchieva Ann Arbor, MI: Michigan Slavic Publications.

\vspace{1mm}

Bylinina, Lisa (2003) O sintaksise otricanija v russkom jazyke: otricatel'nye mestoimenija I konstrukcija `ni... ni...' / On the syntax of negation in Russian: negative polarity pronouns and `ni... ni...' construction. In: Trudy konferencii DIALOG'03. Protvino. 

\vspace{1mm}

Bylinina, Lisa (2002) Balkarskoe otnositelnoe predlo\v{z}enie v tipologi\v{c}eskom osve\v{s}'enii / Relative clause in Balkar in a typological perspective. In: Trudy Kazanskoj \v{s}koly po kompjuternoj I kognitivnoj lingvistike TEL-2002 Kazan: Ote\v{c}estvo, pp. 67-90. 

\end{hangparas}

\vspace{5mm}

\noindent \Large{Teaching} \normalsize{(UTQ certificate acquired Feb '24)}

\vspace{-3mm}
\noindent\line(1,0){250}

\vspace{2mm}
\small{
\noindent\textbf{Winter-Spring 2024 (block 2a)} 

Computational Grammar, 2nd year BA, Information Science, University of Groningen.

BA thesis group supervision, Information Science, University of Groningen.
}

\vspace{2mm}

\vspace{2mm}
\small{
\noindent\textbf{Autumn 2023 (block 1b)} 

Conversational Interfaces: Theory, MA Computer-Mediated Communication, University of Groningen.
}

\vspace{2mm}

\small{
\noindent\textbf{Autumn 2023 (block 1a)} 

Linguistics for Language Technology. 1st year BA, Information Science, University of Groningen.
}

\vspace{2mm}

\small{
\noindent\textbf{Spring 2023 (block 2b)} 

(With Johan Bos)

Conversational Interfaces: Practice. 2nd year MA, Information Science, University of Groningen.
}

\vspace{2mm}

\small{
\noindent\textbf{Spring 2023 (block 2a)} 

(With Gertjan van Noord)

Computational Grammar. 2nd year BA, Information Science, University of Groningen.
}

\vspace{2mm}

\small{
\noindent\textbf{August 2019} 

(With Rick Nouwen)

Numeral Semantics. ESSLLI in Riga, introductory course.
}

\vspace{2mm}

\small{
\noindent\textbf{Autumn 2018}

Semantics I. BA course at Leiden University.
}

\vspace{2mm}

\small{
\noindent\textbf{Autumn 2017}

Semantics I. BA course at Leiden University.
}

\vspace{2mm}

\small{
\noindent\textbf{August 2015}

(With Yasutada Sudo)

The Semantics of Perspective-Sensitivity. ESSLLI 2015 in Barcelona, advanced course.
}

\vspace{2mm}

\small{
\noindent\textbf{Autumn 2014}

(With Lena Karvovskaya)

Semantics I. BA course at Leiden University.
}

\vspace{2mm}

\small{
\noindent\textbf{2005}

(With Igor Yanovich)

The Semantics of Possible and Impossible Worlds. Theoretical and Applied Linguistics Department, Moscow State University.
}

\vspace{6mm}

\noindent \Large{Presentations} \normalsize{(P -- peer-reviewed; I -- upon invitation)}

\vspace{-2mm}
\noindent\line(1,0){300}

\begin{hangparas}{.25in}{1} 

\textbf{P} \hspace{2.5mm}  (with Silvana Abdi, Hylke Brouwer, Martine Elzinga, Shenza Gunput, Sem Huisman, Collin Krooneman, David Poot, Jelmer Top and Cain Weideman)\\
Dutch CoLA: Dutch grammatical knowledge in monolingual and multilingual language models (poster presentation)\\
CLIN 34, Leiden University. June 2024.

\textbf{P} \hspace{2.5mm}  (with Yasutada Sudo and Stavroula Alexandropoulou)\\
Priming acceptability judgments of NPI {\it any}.\\
ELM 3. June 2024.

\textbf{I} \hspace{2.5mm}  Priming acceptability judgments of NPIs\\
University of Trento, Italy. May 2024.

\textbf{I} \hspace{2.5mm}  (with Yasutada Sudo and Stavroula Alexandropoulou)\\
Priming NPI acceptability judgments.\\
ELiTU, Utrecht University. January 2024. 

\textbf{P} \hspace{2.5mm}  (with Alexey Tikhonov and Ekaterina Garmash)\\
Connecting degree and polarity: An artificial language learning study\\
EMNLP main session, Singapore. December 2023.

\textbf{P} \hspace{2.5mm}  (with Natalia Ivlieva and Alexander Podobryaev)\\
Balkar particle `da' and domain maximality\\
WAFL 17, Institute of Mongolian Studies, Ulaanbaatar. September 2023.

\textbf{P} \hspace{2.5mm}  (with Alexey Tikhonov and Denis Paperno)\\
Leverage Points in Modality Shifts: Comparing Language-only and Multimodal Word Representations\\
\ *SEM 2023, co-located with ACL 2023, Toronto. July 2023.

\textbf{I} \hspace{2.5mm}  Invited talk at GroNLP reading group, University of Groningen. September 2022.

\textbf{I} \hspace{2.5mm}  Invited talk at Bar-Ilan University linguistics colloquium. July 2022.

\textbf{I} \hspace{2.08mm} Guest lecture for Models for Language Processing, BSc Artificial Intelligence (advanced course), Utrecht University. June 2022.

\textbf{I} \hspace{2.5mm} Polarity-sensitivity in large language models. Computational Linguistics Seminar. Institute for Logic, Language and Computation, University of Amsterdam. June 2022.

\textbf{P} \hspace{2.5mm}  (with Alexey Tikhonov) \\ Transformers in the loop: Polarity in neural models of language.\\
ACL 2022 main conference. May 2022.

\textbf{P} \hspace{2.5mm}  (with Alexey Tikhonov)\\ Polarity-sensitivity in multilingual pre-trained neural language models.\\
`Jabberwocky Words in Linguistics' workshop. UniBuc \& UMass Amherst. February 2022.

\textbf{I} \hspace{2.5mm}  Polarity in multilingual language models.\\
ILFC Seminar: Interactions between formal and computational linguistics. GDR LIFT, CNRS (online). December 2021.

\textbf{I} \hspace{2.5mm}  Artificial language learning for pre-trained language models.\\
Skolkovo Institute of Science and Technology NLP group seminar. Moscow, October 2021.

\textbf{P} \hspace{2.5mm} (with Alexey Tikhonov and Ekaterina Garmash) \\Artificial Language Learning for Pre-Trained Language Models: \\ Degree Modification and Polarity.\\
`Computational and Experimental Explanations in Semantics and Pragmatics' workshop @ ESSLLI 2021. Utrecht University. August 2021.

\textbf{I} \hspace{2.5mm} Language technology for book recommendations.\\
Language technology night. University of Trento. May 2021.

\textbf{I} \hspace{2.5mm} Artificial language learning for pre-trained language models? On the relation between degree modification and polarity sensitivity. (Response to Stephanie Solt's position talk)\\
Workshop `Scales, degrees and implicature: Novel synergies between Semantics and Pragmatics'.
University of Potsdam. May 2021.

\textbf{I} \hspace{2.5mm} Existence and Plurality.\\
{\bf Keynote} at Sinn und Bedeutung 2020, Queen Mary / UCL. September 2020.

\textbf{P} \hspace{2.5mm} (With Natalia Ivlieva and Alexander Podobryaev)\\
Balkar particle `da' and domain maximality.\\
Sinn und Bedeutung 2020, Queen Mary / UCL. September 2020.

\textbf{I} \hspace{2mm} Numeral Semantics: Quantification, Maximization and Polarity.\\
LaGraM seminar, L'UMR 7023 (SFL), CNRS Pouchet, Paris. November 25 2019.

\textbf{I} \hspace{2mm} Numeral Semantics: Quantification, Maximization and Polarity.\\
Colloquium, Center for Advanced Study in Theoretical Linguistics at the University of Troms\o. October 10 2019.

\textbf{I} \hspace{2mm} Linguistic Typological Databases: Numerals.\\
`Numerals in grammar and beyond' workshop, LUCL, Leiden University. October 8 2019.
 
\textbf{P} \hspace{2mm} (With Stavroula Alexandropoulou and Rick Nouwen) \\
Is there {\em any} licensing in non-DE contexts? An experimental study.\\
Sinn und Bedeutung 2019, Osnabr\"{u}ck University. September 2019.

\textbf{P} \hspace{2mm} (With Stavroula Alexandropoulou and Rick Nouwen) \\
Weak NPIs in non-downward entailing contexts: reasoning about context? (poster)\\
XPRAG 2019, University of Edinburgh. June 2019.

\textbf{I}  \hspace{2mm} What cross-linguistic observations in derivational morphology can tell us about the semantics of cardinals (if anything). \\
Numerals: A Small Workshop, LUCL, Leiden University. June 2019.

\textbf{I}  \hspace{2mm} Numerals and polarity.\\
`The Meaning of Numerals' workshop, ZAS, Berlin. March 2019.

\textbf{I}  \hspace{2mm} Typology of Numerals and The Number Line. \\
{\bf Keynote} at `Typology of Morphosyntactic Parameters' conference, Moscow State University. October 2018.

\textbf{P}  \hspace{2mm} (With Alexander Podobryaev) \\
Number-neutrality and the structure of DPs in Buriat (poster presentation)\\
WAFL 14, MIT, Cambridge MA. October 20 2018.

\textbf{P}  \hspace{2mm} Typology of Numeral Recitation. \\
`Quantity in Language and Thought' workshop, ESSLLI 2018, Sofia, Bulgaria. August 17 2018.

\textbf{I}  \hspace{2mm} (With Alexander Podobryaev) \\
Plurality in Buriat and structurally constrained alternatives. \\
Dialing seminar, Ghent University. April 26 2018.

\textbf{P}  \hspace{2mm} (With Alexander Podobryaev) \\
Plurality in Buriat and structurally constrained alternatives. \\
21st Amsterdam Colloquium, ILLC/Department of Philosophy, University of Amsterdam. December 21 2017.

\vspace{1mm}

\textbf{P}  \hspace{2mm} (With Dominique Blok and Rick Nouwen) \\
A degree quantifier analysis of split scope readings with negative `indefinites'. \\
21st Amsterdam Colloquium. ILLC/Department of Philosophy, University of Amsterdam. December 20 2017.

\vspace{1mm}

\textbf{I} \hspace{2mm} `Zero'. \\
Institut Jean Nicod, ENS. December 6 2017.

\vspace{1mm}

\textbf{I} \hspace{2mm} On the Linguistic Landscape of Subjectivity. \\
Subjective Language Workshop, ZAS, Berlin. November 14 2017.

\vspace{1mm}

\textbf{I} \hspace{2mm}  On typology of subjective adjectives. \\
The Lexicon of Subjectivity Workshop, University of the Basque Country. June 13 2017.

\vspace{1mm}

\textbf{P}  \hspace{2mm} (With Rick Nouwen) \\
The semantics of `zero'.\\
Semantics and Linguistic Theory 27, University of Maryland. May 24 2017.

\vspace{1mm}

\textbf{I} \hspace{2mm} Who is the judge? \\The Third Varieties of Normativity Workshop. Department of Philosophy, Uppsala University. April 24-25 2017.

\vspace{1mm}

\textbf{I} \hspace{2mm} (With Dominique Blok and Rick Nouwen) \\
Splitting Germanic n-words. \\
Leiden-Utrecht Semantics Happenings, Utrecht University. April 13 2017.

\vspace{1mm}

\textbf{I} \hspace{2mm} (With Rick Nouwen) \\ On `zero'. \\ DIP colloquium, ILLC, University of Amsterdam. April 6 2017.

\vspace{1mm}

\textbf{I} \hspace{2mm} (With Sjef Barbiers and Ruby Sleeman) \\ Typology of Numerals and The Number Line. \\Grammar and Cognition Research Group Meeting, ACLC, University of Amsterdam. November 18 2016.

\vspace{1mm}

\textbf{P} \hspace{2mm} (With Heidi Klockmann and Jakub Dotla\v{c}il) \\
Adjectival Modification of Numerals. \\`Logic in Language and in Conversation' workshop, Utrecht University. September 19 2016.

\vspace{1mm}

\textbf{P} \hspace{2mm} (With Annebeth Buis) \\Building an automated pipeline for typological research: 
a case study. \\Dialogue Conference, RSUH, Moscow. June 1-4 2016.

\vspace{1mm}

\textbf{I} \hspace{2mm} Perspective-sensitivity as a type of context-sensitivity. \\`Restriction and Obviation in Scalar Expressions' project seminar, UiL OTS, Utrecht. April 7 2016. 

\vspace{1mm}

\textbf{P} \hspace{1mm} (With Ellen van Drie) \\Typological rarities in the domain of numeral derivation. \\ 11th International Symposium of Cognition, Logic and Communication. Number: Cognitive, Semantic and Crosslinguistic Approaches, University of Latvia, Riga. December 10-11 2015.

\vspace{1mm}

\textbf{P} \hspace{2mm} (With Natalia Ivlieva, Alexander Podobryaev, and Yasutada Sudo) \\ A non-superlative semantics for ordinals.\\ 48th Annual Meeting of the Societas Linguistica Europaea, Leiden University Centre for Linguistics (LUCL), Leiden. September 2-5 2015.

\vspace{1mm}

\textbf{P} \hspace{1.2mm} (With Elin McCready and Yasutada Sudo) \\Perspective-Sensitive Anaphora: The Case of Japanese `Zibun'. \\Semantics and Philosophy in Europe 8, University of Cambridge. 
September 16-19 2015.

\vspace{1mm}

\textbf{P} \hspace{1.2mm} (With Elin McCready and Yasutada Sudo) \\ Perspective-Sensitive Anaphora: The Case of Japanese `Zibun' .\\Pronouns: Syntax, Semantics, Processing: A mini-conference and summer school, HSE, Moscow. June 18 2015.

\vspace{1mm}

\textbf{P} \hspace{1.2mm} (With Elin McCready and Yasutada Sudo) \\The landscape of perspective-shifting. \\Pronouns in Embedded Contexts and the Syntax-Semantics Interface, University of Tuebingen. November 9 2014.

\vspace{1mm}

\textbf{P} \hspace{1.2mm} (With Natalia Ivlieva, Alexander Podobryaev and Yasutada Sudo) \\ A non-superlative semantics for ordinals and the syntax of comparison classes (poster) \\ NELS45, MIT. November 1 2014.

\vspace{1mm}

\textbf{I} \hspace{2mm} Subjectivity, experience, and talking about personal taste. \\Semantics Seminar, Keio University, Tokyo. April 25 2014.

\vspace{1mm}

\textbf{I} \hspace{2mm} The `Language and Number' project Amsterdam/Leiden. \\Meeting of `Les d\'{e}num\'{e}raux \`{a} travers les langues' project, Universit\'{e} Paris Diderot-Paris 7. March 17 2014.

\vspace{1mm}

\textbf{I} \hspace{2mm} Purpose-relativity in the positive degree construction. \\Syntax Seminar, University of Groningen. November 8 2013.

\vspace{1mm}

\textbf{P} \hspace{1.2mm} Distributivity and definiteness in comparison classes (poster) \\ Sinn und Bedeutung 18 in the Basque Country. September 11-13 2013.

\vspace{1mm}

\textbf{P} \hspace{1.2mm} The judge argument (poster) \\ ICL 19 (Congr\`{e}s International des Linguistes), Geneva. July 22-27 2013. 

\vspace{1mm}

\textbf{I} \hspace{1.2mm} Purpose-relativity in degree constructions. \\ Leiden-Utrecht Semantic Happenings (LUSH), Utrecht Institute of Linguistics. June 27 2013.

\vspace{1mm}

\textbf{I} \hspace{2mm} Semantics of subjectivity: `Tasty', `fun' and beyond. \\ Semantics Tea, UCLA. May 8 2013.

\vspace{1mm}

\textbf{P} \hspace{1.2mm} Degree infinitival clauses (poster) \\Semantics and Linguistic Theory 23, UCSC. May 3-5 2013.

\vspace{1mm}

\textbf{I} \hspace{2mm} Degree infinitival clauses. \\Syntax+ seminar, UCS. May 1 2013.

\vspace{1mm}

\textbf{I} \hspace{2mm} Sources of subjectivity and the judge argument. \\ Workshop `The ontology of the mind and the semantics of nominalizations', IHPST, Paris. February 23 2013.

\vspace{1mm}

\textbf{I} \hspace{2mm} Judge-dependence, experience and gradability. \\Seminar 'Structures Formelles du Langage -- UMR 7023', Universit\'{e} Paris VIII. November 5 2012.

\vspace{1mm}

\textbf{P} \hspace{1.2mm} Judge-dependence in degree constructions (poster) \\ NELS 43, CUNY. October 20 2012.

\vspace{1mm}

\textbf{P} \hspace{1.2mm} Judge-dependence in degree constructions. \\ IATL 28, Tel-Aviv University. October 16 2012.

\vspace{1mm}

\textbf{P} \hspace{1mm} Judge-dependence in degree constructions: Evidence from Japanese evidentiality. \\Formal Approaches to Japanese Linguistics 6, ZAS Berlin. September 26 2012.

\vspace{1mm}

\textbf{I} \hspace{2mm} Judge-dependence across degree constructions. \\Guest talk at Galit Sassoon's semantics class, Hebrew University of Jerusalem. April 11 2012.

\vspace{1mm}

\textbf{I} \hspace{2mm} Low degrees and comparative morphology. \\University of Chicago Linguistics Department, Chicago. November 21 2011.

\vspace{1mm}

\textbf{I} \hspace{2mm} Low degrees, comparatives, and how they are related. \\Logical Form Reading Group, MIT. November 4 2011.

\vspace{1mm}

\textbf{P} \hspace{1.2mm} Functional standards and the relative/absolute distinction. \\Sinn und Bedeutung 16, Utrecht. September 6 2011.

\vspace{1mm}

\textbf{P} \hspace{1.2mm} Functional standards. \\ESSLLI Student Session, Ljubljana. August 8 2011.

\vspace{1mm}

\textbf{P} \hspace{1.2mm} Functional standards in Russian and English. \\ FASL20, MIT, Cambridge.  May 13 2011.

\vspace{1mm}

\textbf{P} \hspace{1.2mm} (With Sergei Tatevosov) \\A degree analysis of incrementality revisited. \\`Scalarity in Verb-Based Constructions' workshop, D\"{u}sseldorf. April 7 2011.

\vspace{1mm}

\textbf{P} \hspace{1.2mm} Incremental theme composition in degree semantics. \\Taalkunde in Nederland-dag, Utrecht. February 5 2011.

\vspace{1mm}

\textbf{P} \hspace{1.2mm} It's so NP! \\ 6th International Symposium of Cognition, Logic and Communication, Riga. November 19 2010. 

\vspace{1mm}

\textbf{I} \hspace{2mm} Guest talk on Russian qualitative indefinites. Given at Barbara Partee's seminar in Moscow, Moscow State University. April 24 2009.

\vspace{1mm}

\textbf{P} \hspace{1.2mm} Depreciative Indefinites: Evidence from Russian. \\Formal Descriptions of Slavic Languages 7,5, Moscow. December 6 2008.

\end{hangparas}

\vspace{6mm}

\noindent \Large{Supervision}

\vspace{-4mm}
\noindent\line(1,0){250}

\vspace{2mm}
\small{


\noindent \textbf{14 BA theses supervised} (13 in Groningen, 1 in Utrecht)\\
\noindent \textbf{2 MA theses co-supervised} (1 in Utrecht, 1 in Groningen)\\
\noindent \textbf{4 research assistants and 13 interns supervised} (all part of my VENI project in Leiden)
}

\vspace{6mm}

\noindent \Large{Reviewing}

\vspace{-4mm}
\noindent\line(1,0){250}

\vspace{2mm}

\small{
\begin{hangparas}{.65in}{1} 

\textbf{Journals}: Semantics and Pragmatics (member of the editorial board); Journal of Semantics; Linguistics and Philosophy; Language; Natural Language Semantics; Natural Language and Linguistic Theory; Glossa; Synthese; Journal of Logic, Language and Information; Australasian Journal of Philosophy; Linguistics in the Netherlands; Snippets; Lingua (before 2018).

\end{hangparas}
\vspace{2mm}
\begin{hangparas}{.89in}{1} 

\textbf{Conferences}: ARR, ACL, EMNLP, BlackboxNLP@EMNLP/ACL; GenBench workshop; Semantics and Linguistic Theory; Sinn und Bedeutung; Amsterdam Colloquium; Experiments in Linguistic Meaning (ELM); NELS; XPRAG; GLOW; ConSOLE; ESSLLI Student Session; various ESSLLI workshops.

\end{hangparas}
}

\vspace{5mm}

\noindent \Large{Organizing Experience}

\vspace{-4mm}
\noindent\line(1,0){250}

\vspace{2mm}

\small{
\noindent\textbf{Numerals in grammar and beyond}

(Closing workshop for my VENI project `Number Words')

October 2019}


\vspace{2mm}

\small{
\noindent\textbf{Numerals: A Small Workshop}

https://www.universiteitleiden.nl/en/events/2019/06/numerals-a-small-workshop

LUCL, Leiden University

June 2019}

\vspace{2mm}

\small{
\noindent\textbf{`Leiden-Utrecht Semantics Happenings' talk series}

(Leiden local co-organizer since November 2016)}

\vspace{2mm}

\small{
\noindent\textbf{Reading Group on Numerals}

(January -- November 2016)

The Meertens Institute}

\vspace{2mm}

\small{
\noindent\textbf{Annual Formal Semantics in Moscow workshop}

(Spring 2005 -- Spring 2006)

With Peter Arkadiev, Igor Yanovich and Anna Pazelskaya. 
Took part in organizing FSIM1 and FSIM2 }

\vspace{2mm}

\small{
\noindent\textbf{Formal Semantics Reading Group}

(Spring 2004 -- Spring 2005)

With Igor Yanovich and several other students

Theoretical and Applied Linguistics Department, Moscow State University 
 }

\vspace{5mm}

\noindent \Large{Fieldwork}

\vspace{-4mm}
\noindent\line(1,0){250}

\vspace{2mm}

\small{
%\noindent\textbf{Planned for summer 2024}

%Mongolian, Mongolia and/or Balkar, Kabardino-Balkarija.Topics TBD.
%\vspace{1mm}

\noindent\textbf{Summer 2019}

Balkar, Kabardino-Balkarija. Plurality and quantification.
\vspace{1mm}

Chuvash, Republic of Chuvashia, Russia. Numerals and quantification.


\vspace{2mm}

\small{
\noindent\textbf{Summer 2017}

Buriat, Republic of Buriatia, Russia. Plurality. 

\vspace{2mm}

\small{
\noindent\textbf{Summer 2011}

Tatar, Tatarstan, Russia. Comparative constructions. 

\vspace{2mm}

\small{
\noindent\textbf{Summer 2007}

Ossetian, Northern Ossetia, Russia (Interrupted)

\vspace{2mm}

\noindent\textbf{Summer 2006}

Altai, Altai Republic, Russia. Scalar particles and indefinite pronouns.

\vspace{1mm}

Kabardian, Karachaevo-Cherkessija, Russia. Negation, polarity sensitivity, 
scalar items and indefinite pronouns.

\vspace{2mm}

\noindent\textbf{Summer 2005}

Kabardian, Karachaevo-Cherkessija, Russia. Scalar focus particles and polarity sensitive items.

\vspace{2mm}

\noindent\textbf{Summer 2004}

Adyghe, Republic of Adygea, Russia. Negation and anaphora.

\vspace{1mm}

Kabardian, Karachaevo-Cherkessija, Russia. Anaphora and subordinate constructions.

\vspace{2mm}

\noindent\textbf{Summer 2003}

Bezhta, Dagestan, Russia. Anaphora. (Interrupted)

\vspace{2mm}

\noindent\textbf{Summer 2002}

Balkar, Kabardino-Balkarija, Russia. Syntax of relative clause.
}

\vspace{5mm}

\noindent \Large{Languages}

\vspace{-4mm}
\noindent\line(1,0){250}

\small{
\noindent Natural languages: Russian (native), English (fluent), Dutch (intermediate), French (intermediate). \\
\noindent Programming: Python, Pytorch, Keras, Transformers, Numpy, Pandas, HTML, R, (E)GREP. 

}



\end{document}